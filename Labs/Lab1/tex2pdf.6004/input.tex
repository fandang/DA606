\documentclass[]{article}
\usepackage{lmodern}
\usepackage{amssymb,amsmath}
\usepackage{ifxetex,ifluatex}
\usepackage{fixltx2e} % provides \textsubscript
\ifnum 0\ifxetex 1\fi\ifluatex 1\fi=0 % if pdftex
  \usepackage[T1]{fontenc}
  \usepackage[utf8]{inputenc}
\else % if luatex or xelatex
  \ifxetex
    \usepackage{mathspec}
    \usepackage{xltxtra,xunicode}
  \else
    \usepackage{fontspec}
  \fi
  \defaultfontfeatures{Mapping=tex-text,Scale=MatchLowercase}
  \newcommand{\euro}{€}
\fi
% use upquote if available, for straight quotes in verbatim environments
\IfFileExists{upquote.sty}{\usepackage{upquote}}{}
% use microtype if available
\IfFileExists{microtype.sty}{%
\usepackage{microtype}
\UseMicrotypeSet[protrusion]{basicmath} % disable protrusion for tt fonts
}{}
\usepackage[margin=1in]{geometry}
\usepackage{color}
\usepackage{fancyvrb}
\newcommand{\VerbBar}{|}
\newcommand{\VERB}{\Verb[commandchars=\\\{\}]}
\DefineVerbatimEnvironment{Highlighting}{Verbatim}{commandchars=\\\{\}}
% Add ',fontsize=\small' for more characters per line
\usepackage{framed}
\definecolor{shadecolor}{RGB}{248,248,248}
\newenvironment{Shaded}{\begin{snugshade}}{\end{snugshade}}
\newcommand{\KeywordTok}[1]{\textcolor[rgb]{0.13,0.29,0.53}{\textbf{{#1}}}}
\newcommand{\DataTypeTok}[1]{\textcolor[rgb]{0.13,0.29,0.53}{{#1}}}
\newcommand{\DecValTok}[1]{\textcolor[rgb]{0.00,0.00,0.81}{{#1}}}
\newcommand{\BaseNTok}[1]{\textcolor[rgb]{0.00,0.00,0.81}{{#1}}}
\newcommand{\FloatTok}[1]{\textcolor[rgb]{0.00,0.00,0.81}{{#1}}}
\newcommand{\CharTok}[1]{\textcolor[rgb]{0.31,0.60,0.02}{{#1}}}
\newcommand{\StringTok}[1]{\textcolor[rgb]{0.31,0.60,0.02}{{#1}}}
\newcommand{\CommentTok}[1]{\textcolor[rgb]{0.56,0.35,0.01}{\textit{{#1}}}}
\newcommand{\OtherTok}[1]{\textcolor[rgb]{0.56,0.35,0.01}{{#1}}}
\newcommand{\AlertTok}[1]{\textcolor[rgb]{0.94,0.16,0.16}{{#1}}}
\newcommand{\FunctionTok}[1]{\textcolor[rgb]{0.00,0.00,0.00}{{#1}}}
\newcommand{\RegionMarkerTok}[1]{{#1}}
\newcommand{\ErrorTok}[1]{\textbf{{#1}}}
\newcommand{\NormalTok}[1]{{#1}}
\usepackage{graphicx}
\makeatletter
\def\maxwidth{\ifdim\Gin@nat@width>\linewidth\linewidth\else\Gin@nat@width\fi}
\def\maxheight{\ifdim\Gin@nat@height>\textheight\textheight\else\Gin@nat@height\fi}
\makeatother
% Scale images if necessary, so that they will not overflow the page
% margins by default, and it is still possible to overwrite the defaults
% using explicit options in \includegraphics[width, height, ...]{}
\setkeys{Gin}{width=\maxwidth,height=\maxheight,keepaspectratio}
\ifxetex
  \usepackage[setpagesize=false, % page size defined by xetex
              unicode=false, % unicode breaks when used with xetex
              xetex]{hyperref}
\else
  \usepackage[unicode=true]{hyperref}
\fi
\hypersetup{breaklinks=true,
            bookmarks=true,
            pdfauthor={},
            pdftitle={Introduction to data},
            colorlinks=true,
            citecolor=blue,
            urlcolor=blue,
            linkcolor=magenta,
            pdfborder={0 0 0}}
\urlstyle{same}  % don't use monospace font for urls
\setlength{\parindent}{0pt}
\setlength{\parskip}{6pt plus 2pt minus 1pt}
\setlength{\emergencystretch}{3em}  % prevent overfull lines
\setcounter{secnumdepth}{0}

%%% Use protect on footnotes to avoid problems with footnotes in titles
\let\rmarkdownfootnote\footnote%
\def\footnote{\protect\rmarkdownfootnote}

%%% Change title format to be more compact
\usepackage{titling}

% Create subtitle command for use in maketitle
\newcommand{\subtitle}[1]{
  \posttitle{
    \begin{center}\large#1\end{center}
    }
}

\setlength{\droptitle}{-2em}
  \title{Introduction to data}
  \pretitle{\vspace{\droptitle}\centering\huge}
  \posttitle{\par}
  \author{}
  \preauthor{}\postauthor{}
  \date{}
  \predate{}\postdate{}



\begin{document}

\maketitle


Some define Statistics as the field that focuses on turning information
into knowledge. The first step in that process is to summarize and
describe the raw information - the data. In this lab, you will gain
insight into public health by generating simple graphical and numerical
summaries of a data set collected by the Centers for Disease Control and
Prevention (CDC). As this is a large data set, along the way you'll also
learn the indispensable skills of data processing and subsetting.

\subsection{Getting started}\label{getting-started}

The Behavioral Risk Factor Surveillance System (BRFSS) is an annual
telephone survey of 350,000 people in the United States. As its name
implies, the BRFSS is designed to identify risk factors in the adult
population and report emerging health trends. For example, respondents
are asked about their diet and weekly physical activity, their HIV/AIDS
status, possible tobacco use, and even their level of healthcare
coverage. The BRFSS Web site
(\href{http://www.cdc.gov/brfss}{\url{http://www.cdc.gov/brfss}})
contains a complete description of the survey, including the research
questions that motivate the study and many interesting results derived
from the data.

We will focus on a random sample of 20,000 people from the BRFSS survey
conducted in 2000. While there are over 200 variables in this data set,
we will work with a small subset.

We begin by loading the data set of 20,000 observations into the R
workspace. After launching RStudio, enter the following command.

\begin{Shaded}
\begin{Highlighting}[]
\KeywordTok{source}\NormalTok{(}\StringTok{"more/cdc.R"}\NormalTok{)}
\end{Highlighting}
\end{Shaded}

The data set \texttt{cdc} that shows up in your workspace is a
\emph{data matrix}, with each row representing a \emph{case} and each
column representing a \emph{variable}. R calls this data format a
\emph{data frame}, which is a term that will be used throughout the
labs.

To view the names of the variables, type the command

\begin{Shaded}
\begin{Highlighting}[]
\KeywordTok{names}\NormalTok{(cdc)}
\end{Highlighting}
\end{Shaded}

This returns the names \texttt{genhlth}, \texttt{exerany},
\texttt{hlthplan}, \texttt{smoke100}, \texttt{height}, \texttt{weight},
\texttt{wtdesire}, \texttt{age}, and \texttt{gender}. Each one of these
variables corresponds to a question that was asked in the survey. For
example, for \texttt{genhlth}, respondents were asked to evaluate their
general health, responding either excellent, very good, good, fair or
poor. The \texttt{exerany} variable indicates whether the respondent
exercised in the past month (1) or did not (0). Likewise,
\texttt{hlthplan} indicates whether the respondent had some form of
health coverage (1) or did not (0). The \texttt{smoke100} variable
indicates whether the respondent had smoked at least 100 cigarettes in
her lifetime. The other variables record the respondent's
\texttt{height} in inches, \texttt{weight} in pounds as well as their
desired weight, \texttt{wtdesire}, \texttt{age} in years, and
\texttt{gender}.

\begin{enumerate}
\def\labelenumi{\arabic{enumi}.}
\itemsep1pt\parskip0pt\parsep0pt
\item
  How many cases are there in this data set? How many variables? For
  each variable, identify its data type (e.g.~categorical, discrete).
\end{enumerate}

\begin{Shaded}
\begin{Highlighting}[]
\KeywordTok{source}\NormalTok{(}\StringTok{"more/cdc.R"}\NormalTok{)}

\NormalTok{numRows <-}\StringTok{ }\KeywordTok{nrow}\NormalTok{(cdc)}
\KeywordTok{cat}\NormalTok{(}\StringTok{'Num Cases: '}\NormalTok{,numRows)}
\end{Highlighting}
\end{Shaded}

\begin{verbatim}
## Num Cases:  20000
\end{verbatim}

\begin{Shaded}
\begin{Highlighting}[]
\NormalTok{numVars <-}\StringTok{ }\KeywordTok{ncol}\NormalTok{(cdc)}
\KeywordTok{cat}\NormalTok{(}\StringTok{'Num Variables: '}\NormalTok{,numVars)}
\end{Highlighting}
\end{Shaded}

\begin{verbatim}
## Num Variables:  9
\end{verbatim}

\begin{Shaded}
\begin{Highlighting}[]
\KeywordTok{str}\NormalTok{(cdc)}
\end{Highlighting}
\end{Shaded}

\begin{verbatim}
## 'data.frame':    20000 obs. of  9 variables:
##  $ genhlth : Factor w/ 5 levels "excellent","very good",..: 3 3 3 3 2 2 2 2 3 3 ...
##  $ exerany : num  0 0 1 1 0 1 1 0 0 1 ...
##  $ hlthplan: num  1 1 1 1 1 1 1 1 1 1 ...
##  $ smoke100: num  0 1 1 0 0 0 0 0 1 0 ...
##  $ height  : num  70 64 60 66 61 64 71 67 65 70 ...
##  $ weight  : int  175 125 105 132 150 114 194 170 150 180 ...
##  $ wtdesire: int  175 115 105 124 130 114 185 160 130 170 ...
##  $ age     : int  77 33 49 42 55 55 31 45 27 44 ...
##  $ gender  : Factor w/ 2 levels "m","f": 1 2 2 2 2 2 1 1 2 1 ...
\end{verbatim}

\textbf{genhlth : Categorical+Ordinal}

\textbf{exerany : Numerical+Discrete}

\textbf{hlthplan: Categorical+Regular}

\textbf{smoke100: Categorical+Regular}

\textbf{height : Numerical+Discrete}

\textbf{weight : Numerical+Discrete}

\textbf{wtdesire: Numerical+Discrete}

\textbf{age : Numerical+Discrete}

\textbf{gender : Categorical+Regular}

We can have a look at the first few entries (rows) of our data with the
command

\begin{Shaded}
\begin{Highlighting}[]
\KeywordTok{head}\NormalTok{(cdc)}
\end{Highlighting}
\end{Shaded}

and similarly we can look at the last few by typing

\begin{Shaded}
\begin{Highlighting}[]
\KeywordTok{tail}\NormalTok{(cdc)}
\end{Highlighting}
\end{Shaded}

You could also look at \emph{all} of the data frame at once by typing
its name into the console, but that might be unwise here. We know
\texttt{cdc} has 20,000 rows, so viewing the entire data set would mean
flooding your screen. It's better to take small peeks at the data with
\texttt{head}, \texttt{tail} or the subsetting techniques that you'll
learn in a moment.

\subsection{Summaries and tables}\label{summaries-and-tables}

The BRFSS questionnaire is a massive trove of information. A good first
step in any analysis is to distill all of that information into a few
summary statistics and graphics. As a simple example, the function
\texttt{summary} returns a numerical summary: minimum, first quartile,
median, mean, second quartile, and maximum. For \texttt{weight} this is

\begin{Shaded}
\begin{Highlighting}[]
\KeywordTok{summary}\NormalTok{(cdc$weight)}
\end{Highlighting}
\end{Shaded}

R also functions like a very fancy calculator. If you wanted to compute
the interquartile range for the respondents' weight, you would look at
the output from the summary command above and then enter

\begin{Shaded}
\begin{Highlighting}[]
\DecValTok{190} \NormalTok{-}\StringTok{ }\DecValTok{140}
\end{Highlighting}
\end{Shaded}

R also has built-in functions to compute summary statistics one by one.
For instance, to calculate the mean, median, and variance of
\texttt{weight}, type

\begin{Shaded}
\begin{Highlighting}[]
\KeywordTok{mean}\NormalTok{(cdc$weight) }
\KeywordTok{var}\NormalTok{(cdc$weight)}
\KeywordTok{median}\NormalTok{(cdc$weight)}
\end{Highlighting}
\end{Shaded}

While it makes sense to describe a quantitative variable like
\texttt{weight} in terms of these statistics, what about categorical
data? We would instead consider the sample frequency or relative
frequency distribution. The function \texttt{table} does this for you by
counting the number of times each kind of response was given. For
example, to see the number of people who have smoked 100 cigarettes in
their lifetime, type

\begin{Shaded}
\begin{Highlighting}[]
\KeywordTok{table}\NormalTok{(cdc$smoke100)}
\end{Highlighting}
\end{Shaded}

or instead look at the relative frequency distribution by typing

\begin{Shaded}
\begin{Highlighting}[]
\KeywordTok{table}\NormalTok{(cdc$smoke100)/}\DecValTok{20000}
\end{Highlighting}
\end{Shaded}

Notice how R automatically divides all entries in the table by 20,000 in
the command above. This is similar to something we observed in the
Introduction to R; when we multiplied or divided a vector with a number,
R applied that action across entries in the vectors. As we see above,
this also works for tables. Next, we make a bar plot of the entries in
the table by putting the table inside the \texttt{barplot} command.

\begin{Shaded}
\begin{Highlighting}[]
\KeywordTok{barplot}\NormalTok{(}\KeywordTok{table}\NormalTok{(cdc$smoke100))}
\end{Highlighting}
\end{Shaded}

Notice what we've done here! We've computed the table of
\texttt{cdc\$smoke100} and then immediately applied the graphical
function, \texttt{barplot}. This is an important idea: R commands can be
nested. You could also break this into two steps by typing the
following:

\begin{Shaded}
\begin{Highlighting}[]
\NormalTok{smoke <-}\StringTok{ }\KeywordTok{table}\NormalTok{(cdc$smoke100)}

\KeywordTok{barplot}\NormalTok{(smoke)}
\end{Highlighting}
\end{Shaded}

Here, we've made a new object, a table, called \texttt{smoke} (the
contents of which we can see by typing \texttt{smoke} into the console)
and then used it in as the input for \texttt{barplot}. The special
symbol \texttt{\textless{}-} performs an \emph{assignment}, taking the
output of one line of code and saving it into an object in your
workspace. This is another important idea that we'll return to later.

\begin{enumerate}
\def\labelenumi{\arabic{enumi}.}
\setcounter{enumi}{1}
\itemsep1pt\parskip0pt\parsep0pt
\item
  Create a numerical summary for \texttt{height} and \texttt{age}, and
  compute the interquartile range for each. Compute the relative
  frequency distribution for \texttt{gender} and \texttt{exerany}. How
  many males are in the sample? What proportion of the sample reports
  being in excellent health?
\end{enumerate}

\begin{Shaded}
\begin{Highlighting}[]
\KeywordTok{summary}\NormalTok{(cdc$height)}
\end{Highlighting}
\end{Shaded}

\begin{verbatim}
##    Min. 1st Qu.  Median    Mean 3rd Qu.    Max. 
##   48.00   64.00   67.00   67.18   70.00   93.00
\end{verbatim}

\begin{Shaded}
\begin{Highlighting}[]
\KeywordTok{summary}\NormalTok{(cdc$age)}
\end{Highlighting}
\end{Shaded}

\begin{verbatim}
##    Min. 1st Qu.  Median    Mean 3rd Qu.    Max. 
##   18.00   31.00   43.00   45.07   57.00   99.00
\end{verbatim}

\begin{Shaded}
\begin{Highlighting}[]
\KeywordTok{IQR}\NormalTok{(cdc$height)}
\end{Highlighting}
\end{Shaded}

\begin{verbatim}
## [1] 6
\end{verbatim}

\begin{Shaded}
\begin{Highlighting}[]
\KeywordTok{IQR}\NormalTok{(cdc$age)}
\end{Highlighting}
\end{Shaded}

\begin{verbatim}
## [1] 26
\end{verbatim}

relative frequency distribution for gender

\begin{Shaded}
\begin{Highlighting}[]
\KeywordTok{table}\NormalTok{(cdc$gender) /}\StringTok{ }\KeywordTok{nrow}\NormalTok{(cdc)}
\end{Highlighting}
\end{Shaded}

\begin{verbatim}
## 
##       m       f 
## 0.47845 0.52155
\end{verbatim}

relative frequency distribution for exerany

\begin{Shaded}
\begin{Highlighting}[]
\KeywordTok{table}\NormalTok{(cdc$exerany) /}\StringTok{ }\KeywordTok{nrow}\NormalTok{(cdc)}
\end{Highlighting}
\end{Shaded}

\begin{verbatim}
## 
##      0      1 
## 0.2543 0.7457
\end{verbatim}

The \texttt{table} command can be used to tabulate any number of
variables that you provide. For example, to examine which participants
have smoked across each gender, we could use the following.

\begin{Shaded}
\begin{Highlighting}[]
\KeywordTok{table}\NormalTok{(cdc$gender,cdc$smoke100)}
\end{Highlighting}
\end{Shaded}

Here, we see column labels of 0 and 1. Recall that 1 indicates a
respondent has smoked at least 100 cigarettes. The rows refer to gender.
To create a mosaic plot of this table, we would enter the following
command.

\begin{Shaded}
\begin{Highlighting}[]
\KeywordTok{mosaicplot}\NormalTok{(}\KeywordTok{table}\NormalTok{(cdc$gender,cdc$smoke100))}
\end{Highlighting}
\end{Shaded}

We could have accomplished this in two steps by saving the table in one
line and applying \texttt{mosaicplot} in the next (see the table/barplot
example above).

\begin{enumerate}
\def\labelenumi{\arabic{enumi}.}
\setcounter{enumi}{2}
\itemsep1pt\parskip0pt\parsep0pt
\item
  What does the mosaic plot reveal about smoking habits and gender?
\end{enumerate}

\subsection{Interlude: How R thinks about
data}\label{interlude-how-r-thinks-about-data}

We mentioned that R stores data in data frames, which you might think of
as a type of spreadsheet. Each row is a different observation (a
different respondent) and each column is a different variable (the first
is \texttt{genhlth}, the second \texttt{exerany} and so on). We can see
the size of the data frame next to the object name in the workspace or
we can type

\begin{Shaded}
\begin{Highlighting}[]
\KeywordTok{dim}\NormalTok{(cdc)}
\end{Highlighting}
\end{Shaded}

which will return the number of rows and columns. Now, if we want to
access a subset of the full data frame, we can use row-and-column
notation. For example, to see the sixth variable of the 567th
respondent, use the format

\begin{Shaded}
\begin{Highlighting}[]
\NormalTok{cdc[}\DecValTok{567}\NormalTok{,}\DecValTok{6}\NormalTok{]}
\end{Highlighting}
\end{Shaded}

which means we want the element of our data set that is in the 567th row
(meaning the 567th person or observation) and the 6th column (in this
case, weight). We know that \texttt{weight} is the 6th variable because
it is the 6th entry in the list of variable names

\begin{Shaded}
\begin{Highlighting}[]
\KeywordTok{names}\NormalTok{(cdc)}
\end{Highlighting}
\end{Shaded}

To see the weights for the first 10 respondents we can type

\begin{Shaded}
\begin{Highlighting}[]
\NormalTok{cdc[}\DecValTok{1}\NormalTok{:}\DecValTok{10}\NormalTok{,}\DecValTok{6}\NormalTok{]}
\end{Highlighting}
\end{Shaded}

In this expression, we have asked just for rows in the range 1 through
10. R uses the \texttt{:} to create a range of values, so 1:10 expands
to 1, 2, 3, 4, 5, 6, 7, 8, 9, 10. You can see this by entering

\begin{Shaded}
\begin{Highlighting}[]
\DecValTok{1}\NormalTok{:}\DecValTok{10}
\end{Highlighting}
\end{Shaded}

Finally, if we want all of the data for the first 10 respondents, type

\begin{Shaded}
\begin{Highlighting}[]
\NormalTok{cdc[}\DecValTok{1}\NormalTok{:}\DecValTok{10}\NormalTok{,]}
\end{Highlighting}
\end{Shaded}

By leaving out an index or a range (we didn't type anything between the
comma and the square bracket), we get all the columns. When starting out
in R, this is a bit counterintuitive. As a rule, we omit the column
number to see all columns in a data frame. Similarly, if we leave out an
index or range for the rows, we would access all the observations, not
just the 567th, or rows 1 through 10. Try the following to see the
weights for all 20,000 respondents fly by on your screen

\begin{Shaded}
\begin{Highlighting}[]
\NormalTok{cdc[,}\DecValTok{6}\NormalTok{]}
\end{Highlighting}
\end{Shaded}

Recall that column 6 represents respondents' weight, so the command
above reported all of the weights in the data set. An alternative method
to access the weight data is by referring to the name. Previously, we
typed \texttt{names(cdc)} to see all the variables contained in the cdc
data set. We can use any of the variable names to select items in our
data set.

\begin{Shaded}
\begin{Highlighting}[]
\NormalTok{cdc$weight}
\end{Highlighting}
\end{Shaded}

The dollar-sign tells R to look in data frame \texttt{cdc} for the
column called \texttt{weight}. Since that's a single vector, we can
subset it with just a single index inside square brackets. We see the
weight for the 567th respondent by typing

\begin{Shaded}
\begin{Highlighting}[]
\NormalTok{cdc$weight[}\DecValTok{567}\NormalTok{]}
\end{Highlighting}
\end{Shaded}

Similarly, for just the first 10 respondents

\begin{Shaded}
\begin{Highlighting}[]
\NormalTok{cdc$weight[}\DecValTok{1}\NormalTok{:}\DecValTok{10}\NormalTok{]}
\end{Highlighting}
\end{Shaded}

The command above returns the same result as the
\texttt{cdc{[}1:10,6{]}} command. Both row-and-column notation and
dollar-sign notation are widely used, which one you choose to use
depends on your personal preference.

\subsection{A little more on
subsetting}\label{a-little-more-on-subsetting}

It's often useful to extract all individuals (cases) in a data set that
have specific characteristics. We accomplish this through
\emph{conditioning} commands. First, consider expressions like

\begin{Shaded}
\begin{Highlighting}[]
\NormalTok{cdc$gender ==}\StringTok{ "m"}
\end{Highlighting}
\end{Shaded}

or

\begin{Shaded}
\begin{Highlighting}[]
\NormalTok{cdc$age >}\StringTok{ }\DecValTok{30}
\end{Highlighting}
\end{Shaded}

These commands produce a series of \texttt{TRUE} and \texttt{FALSE}
values. There is one value for each respondent, where \texttt{TRUE}
indicates that the person was male (via the first command) or older than
30 (second command).

Suppose we want to extract just the data for the men in the sample, or
just for those over 30. We can use the R function \texttt{subset} to do
that for us. For example, the command

\begin{Shaded}
\begin{Highlighting}[]
\NormalTok{mdata <-}\StringTok{ }\KeywordTok{subset}\NormalTok{(cdc, cdc$gender ==}\StringTok{ "m"}\NormalTok{)}
\end{Highlighting}
\end{Shaded}

will create a new data set called \texttt{mdata} that contains only the
men from the \texttt{cdc} data set. In addition to finding it in your
workspace alongside its dimensions, you can take a peek at the first
several rows as usual

\begin{Shaded}
\begin{Highlighting}[]
\KeywordTok{head}\NormalTok{(mdata)}
\end{Highlighting}
\end{Shaded}

This new data set contains all the same variables but just under half
the rows. It is also possible to tell R to keep only specific variables,
which is a topic we'll discuss in a future lab. For now, the important
thing is that we can carve up the data based on values of one or more
variables.

As an aside, you can use several of these conditions together with
\texttt{\&} and \texttt{\textbar{}}. The \texttt{\&} is read ``and'' so
that

\begin{Shaded}
\begin{Highlighting}[]
\NormalTok{m_and_over30 <-}\StringTok{ }\KeywordTok{subset}\NormalTok{(cdc, gender ==}\StringTok{ "m"} \NormalTok{&}\StringTok{ }\NormalTok{age >}\StringTok{ }\DecValTok{30}\NormalTok{)}
\end{Highlighting}
\end{Shaded}

will give you the data for men over the age of 30. The
\texttt{\textbar{}} character is read ``or'' so that

\begin{Shaded}
\begin{Highlighting}[]
\NormalTok{m_or_over30 <-}\StringTok{ }\KeywordTok{subset}\NormalTok{(cdc, gender ==}\StringTok{ "m"} \NormalTok{|}\StringTok{ }\NormalTok{age >}\StringTok{ }\DecValTok{30}\NormalTok{)}
\end{Highlighting}
\end{Shaded}

will take people who are men or over the age of 30 (why that's an
interesting group is hard to say, but right now the mechanics of this
are the important thing). In principle, you may use as many ``and'' and
``or'' clauses as you like when forming a subset.

\begin{enumerate}
\def\labelenumi{\arabic{enumi}.}
\setcounter{enumi}{2}
\itemsep1pt\parskip0pt\parsep0pt
\item
  Create a new object called \texttt{under23\_and\_smoke} that contains
  all observations of respondents under the age of 23 that have smoked
  100 cigarettes in their lifetime. Write the command you used to create
  the new object as the answer to this exercise.
\end{enumerate}

\subsection{Quantitative data}\label{quantitative-data}

With our subsetting tools in hand, we'll now return to the task of the
day: making basic summaries of the BRFSS questionnaire. We've already
looked at categorical data such as \texttt{smoke} and \texttt{gender} so
now let's turn our attention to quantitative data. Two common ways to
visualize quantitative data are with box plots and histograms. We can
construct a box plot for a single variable with the following command.

\begin{Shaded}
\begin{Highlighting}[]
\KeywordTok{boxplot}\NormalTok{(cdc$height)}
\end{Highlighting}
\end{Shaded}

You can compare the locations of the components of the box by examining
the summary statistics.

\begin{Shaded}
\begin{Highlighting}[]
\KeywordTok{summary}\NormalTok{(cdc$height)}
\end{Highlighting}
\end{Shaded}

Confirm that the median and upper and lower quartiles reported in the
numerical summary match those in the graph. The purpose of a boxplot is
to provide a thumbnail sketch of a variable for the purpose of comparing
across several categories. So we can, for example, compare the heights
of men and women with

\begin{Shaded}
\begin{Highlighting}[]
\KeywordTok{boxplot}\NormalTok{(cdc$height ~}\StringTok{ }\NormalTok{cdc$gender)}
\end{Highlighting}
\end{Shaded}

The notation here is new. The \texttt{\textasciitilde{}} character can
be read \emph{versus} or \emph{as a function of}. So we're asking R to
give us a box plots of heights where the groups are defined by gender.

Next let's consider a new variable that doesn't show up directly in this
data set: Body Mass Index (BMI)
(\href{http://en.wikipedia.org/wiki/Body_mass_index}{\url{http://en.wikipedia.org/wiki/Body_mass_index}}).
BMI is a weight to height ratio and can be calculated as:

\[ BMI = \frac{weight~(lb)}{height~(in)^2} * 703 \]

703 is the approximate conversion factor to change units from metric
(meters and kilograms) to imperial (inches and pounds).

The following two lines first make a new object called \texttt{bmi} and
then creates box plots of these values, defining groups by the variable
\texttt{cdc\$genhlth}.

\begin{Shaded}
\begin{Highlighting}[]
\NormalTok{bmi <-}\StringTok{ }\NormalTok{(cdc$weight /}\StringTok{ }\NormalTok{cdc$height^}\DecValTok{2}\NormalTok{) *}\StringTok{ }\DecValTok{703}
\KeywordTok{boxplot}\NormalTok{(bmi ~}\StringTok{ }\NormalTok{cdc$genhlth)}
\end{Highlighting}
\end{Shaded}

Notice that the first line above is just some arithmetic, but it's
applied to all 20,000 numbers in the \texttt{cdc} data set. That is, for
each of the 20,000 participants, we take their weight, divide by their
height-squared and then multiply by 703. The result is 20,000 BMI
values, one for each respondent. This is one reason why we like R: it
lets us perform computations like this using very simple expressions.

\begin{enumerate}
\def\labelenumi{\arabic{enumi}.}
\setcounter{enumi}{3}
\itemsep1pt\parskip0pt\parsep0pt
\item
  What does this box plot show? Pick another categorical variable from
  the data set and see how it relates to BMI. List the variable you
  chose, why you might think it would have a relationship to BMI, and
  indicate what the figure seems to suggest.
\end{enumerate}

Finally, let's make some histograms. We can look at the histogram for
the age of our respondents with the command

\begin{Shaded}
\begin{Highlighting}[]
\KeywordTok{hist}\NormalTok{(cdc$age)}
\end{Highlighting}
\end{Shaded}

Histograms are generally a very good way to see the shape of a single
distribution, but that shape can change depending on how the data is
split between the different bins. You can control the number of bins by
adding an argument to the command. In the next two lines, we first make
a default histogram of \texttt{bmi} and then one with 50 breaks.

\begin{Shaded}
\begin{Highlighting}[]
\KeywordTok{hist}\NormalTok{(bmi)}
\KeywordTok{hist}\NormalTok{(bmi, }\DataTypeTok{breaks =} \DecValTok{50}\NormalTok{)}
\end{Highlighting}
\end{Shaded}

Note that you can flip between plots that you've created by clicking the
forward and backward arrows in the lower right region of RStudio, just
above the plots. How do these two histograms compare?

At this point, we've done a good first pass at analyzing the information
in the BRFSS questionnaire. We've found an interesting association
between smoking and gender, and we can say something about the
relationship between people's assessment of their general health and
their own BMI. We've also picked up essential computing tools -- summary
statistics, subsetting, and plots -- that will serve us well throughout
this course.

\begin{center}\rule{0.5\linewidth}{\linethickness}\end{center}

\subsection{On Your Own}\label{on-your-own}

\begin{itemize}
\item
  Make a scatterplot of weight versus desired weight. Describe the
  relationship between these two variables.
\item
  Let's consider a new variable: the difference between desired weight
  (\texttt{wtdesire}) and current weight (\texttt{weight}). Create this
  new variable by subtracting the two columns in the data frame and
  assigning them to a new object called \texttt{wdiff}.
\item
  What type of data is \texttt{wdiff}? If an observation \texttt{wdiff}
  is 0, what does this mean about the person's weight and desired
  weight. What if \texttt{wdiff} is positive or negative?
\item
  Describe the distribution of \texttt{wdiff} in terms of its center,
  shape, and spread, including any plots you use. What does this tell us
  about how people feel about their current weight?
\item
  Using numerical summaries and a side-by-side box plot, determine if
  men tend to view their weight differently than women.
\item
  Now it's time to get creative. Find the mean and standard deviation of
  \texttt{weight} and determine what proportion of the weights are
  within one standard deviation of the mean.
\end{itemize}

\hyperdef{}{license}{}
This is a product of OpenIntro that is released under a
\href{http://creativecommons.org/licenses/by-sa/3.0}{Creative Commons
Attribution-ShareAlike 3.0 Unported}. This lab was adapted for OpenIntro
by Andrew Bray and Mine Çetinkaya-Rundel from a lab written by Mark
Hansen of UCLA Statistics.

\end{document}
